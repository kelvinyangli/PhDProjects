%%%%%%%%%%%%%%%%%%%%%%% file template.tex %%%%%%%%%%%%%%%%%%%%%%%%%
%
% This is a general template file for the LaTeX package SVJour3
% for Springer journals.          Springer Heidelberg 2010/09/16
%
% Copy it to a new file with a new name and use it as the basis
% for your article. Delete % signs as needed.
%
% This template includes a few options for different layouts and
% content for various journals. Please consult a previous issue of
% your journal as needed.
%
%%%%%%%%%%%%%%%%%%%%%%%%%%%%%%%%%%%%%%%%%%%%%%%%%%%%%%%%%%%%%%%%%%%
%
% First comes an example EPS file -- just ignore it and
% proceed on the \documentclass line
% your LaTeX will extract the file if required
\begin{filecontents*}{example.eps}
%!PS-Adobe-3.0 EPSF-3.0
%%BoundingBox: 19 19 221 221
%%CreationDate: Mon Sep 29 1997
%%Creator: programmed by hand (JK)
%%EndComments
gsave
newpath
  20 20 moveto
  20 220 lineto
  220 220 lineto
  220 20 lineto
closepath
2 setlinewidth
gsave
  .4 setgray fill
grestore
stroke
grestore
\end{filecontents*}
%
\RequirePackage{fix-cm}
%
%\documentclass{svjour3}                     % onecolumn (standard format)
%\documentclass[smallcondensed]{svjour3}     % onecolumn (ditto)
\documentclass[smallextended]{svjour3}       % onecolumn (second format)
%\documentclass[twocolumn]{svjour3}          % twocolumn
%
\smartqed  % flush right qed marks, e.g. at end of proof
%
% \usepackage{mathptmx}      % use Times fonts if available on your TeX system
%
% insert here the call for the packages your document requires
\usepackage{latexsym}
\usepackage{graphicx}
\usepackage{amsmath, amssymb, amsfonts, mathtools}
\usepackage{enumerate}
\usepackage{algorithm, algpseudocode}
\usepackage{txfonts, pxfonts}
\usepackage{grffile}
\usepackage{caption, subcaption}
\usepackage{listings}
\usepackage[table, xcdraw]{xcolor}
\usepackage{rotating}
\usepackage{multirow}
\usepackage{chronology}
\usetikzlibrary{arrows, shapes}
\usepackage{tabularx}
\usepackage{hyperref}
\usepackage{libertine}
\usepackage{pgfgantt}
\usepackage{lscape}
\usepackage{enumitem}
\input amssym.def
\input amssym.tex
%
% please place your own definitions here and don't use \def but
\newcommand{\mmlcpt}{$mbMML_{CPT}$ }
\newcommand{\mbptmml}{$MBPT_{mml}$ }
\DeclareMathOperator*{\argmin}{arg\,min}
\DeclarePairedDelimiter\floor{\lfloor}{\rfloor}
\newcommand{\ci}{\mathrel{\text{\scalebox{1.07}{$\perp\mkern-10mu\perp$}}}}
\newcommand{\independent}{\perp\mkern-9.5mu\perp}
\newcommand{\notindependent}{\centernot{\independent}}
\newcommand{\qedwhite}{\hfill \ensuremath{\Box}}

\makeatletter
% Taken from http://ctan.org/pkg/centernot
\newcommand*{\centernot}{%
  \mathpalette\@centernot
}
\def\@centernot#1#2{%
  \mathrel{%
    \rlap{%
      \settowidth\dimen@{$\m@th#1{#2}$}%
      \kern.5\dimen@
      \settowidth\dimen@{$\m@th#1=$}%
      \kern-.5\dimen@
      $\m@th#1\not$%
    }%
    {#2}%
  }%
}
\makeatother

%
% Insert the name of "your journal" with
% \journalname{myjournal}
%
% This file can be modified and used in other conferences as long
% as credit to the authors and supporting agencies is retained, this notice
% is not changed, and further modification or reuse is not restricted.

\begin{document}

\title{Related definitions and notations%\thanks{Grants or other notes
%about the article that should go on the front page should be
%placed here. General acknowledgments should be placed at the end of the article.}
}
%\subtitle{Do you have a subtitle?\\ If so, write it here}

%\titlerunning{Short form of title}        % if too long for running head

\author{First Author         \and
        Second Author \and
        Third Author
}

%\authorrunning{Short form of author list} % if too long for running head

\institute{F. Author \at
              first address \\
              Tel.: +123-45-678910\\
              Fax: +123-45-678910\\
              \email{fauthor@example.com}           %  \\
%             \emph{Present address:} of F. Author  %  if needed
           \and%
           S. Author \at
              second address
}

\date{Received: date / Accepted: date}
% The correct dates will be entered by the editor

\maketitle

The graph theory definitions are adopted from \cite{diestel2000graph}. The Bayesian network definitions and notations are largely adopted from \cite{neapolitan2004learning}. 

\section{Probability theory}
\begin{itemize}
\item marginal probability distribution
\item joint probability distribution
\item conditional probability distribution
\item Bayes' Theorem
\item independent
\item conditional independent
\item prior propability
\item posterior probability
\end{itemize}

\section{Graphical model}
\begin{itemize}
\item graph - nodes, vertices, variables are used interchangably; edges, arcs, links are used interchangably;
\item 
\end{itemize}

\begin{definition}
\label{def:graph}
A \textbf{graph} is a pair $G = (V, E)$ comprising a set $V$ of vertices (or nodes) together with a set $E$ of edges (or arcs).
\end{definition}
The vertex set of a graph $G$ is referred to as $V(G)$, its edge set as $E(G)$. 

\begin{definition}
\label{def:neighbour}
Two vertices $x, y$ of $G$ are \textbf{adjacent}, or \textbf{neighbours}, if $xy$ is an edge of $G$. 
\end{definition}

\begin{definition}
\label{def:subgraph}
Let $G=(V,E)$ and $G'=(V',E')$ be two graphs. If $V' \subseteq V$ and $E' \subseteq E$, then $G'$ is a \textbf{subgraph} of $G$ (and $G$ is a \textbf{supergraph} of $G'$), written as $G' \subseteq G$. If $G' \subseteq G$ and $G'$ contains all the edges $xy \in E$ with $x, y \in V'$, then $G'$ is an \textbf{induced subgraph} of $G$, written as $G' = G[V']$. 
\end{definition}

\begin{definition}
\label{def:path}
A \textbf{path} in a graph $G$ is a non-empty graph $P = (V,E)$ of the form 
\begin{align*}
V = \{X_0, X_1, \dots, X_k\} \text{,    } E = \{X_0X_1, X_1X_2, \dots, X_{k-1}X_k\},
\end{align*}
where the $X_i$s are all distinct. If $k \ge 3$ and $X_k =X_0$, then the graph $C = (V,E)$ is called a \textbf{cycle}.
\end{definition}

\begin{definition}
\label{def:connected}
A non-empty graph $G$ is called \textbf{connected} if any two of its vertices are linked by a path in $G$. 
\end{definition}

\begin{definition}
\label{def:tree}
An acyclic graph, one not containing any cycles, is called a \textbf{forest}. A connected forest is called a \textbf{tree}. 
\end{definition}

\begin{definition}
\label{def:digraph}
A \textbf{directed graph} (or \textbf{digraph}) is a pair $(V,E)$, where $V$ is a finite set of vertices, and $E$ is a set of ordered pairs of distinct vertices in $V$.
\end{definition}

\begin{definition}
\label{def:dag}
A directed graph $G = (V, E)$ is called a directed acyclic graph (\textbf{DAG}) if it contains no directed cycles. Given nodes $X_1$ and $X_2$ in $V$, $X_1$ is called a \textbf{parent} of $X_2$ and $X_2$ is called a \textbf{child} of $X_1$ if there is a directed edge from $X_1$ to $X_2$. $X_1$ is called an \textbf{ancestor} of $X_2$ and $X_2$ is called a \textbf{descendent} of $X_1$ if there is a directed path from $X_1$ to $X_2$. $X_2$ is a \textbf{nondescendent} of $X_1$ if $X_2$ is not a descendent of $X_1$.
\end{definition}

\begin{definition}
A \textbf{chain graph} \cite{frydenberg1990chain} is a graph which may have directed and undirected edges, but with no directed cycles.  
\end{definition}
Undirected graphs and DAGs are special cases of chain graphs. The definition was later extended to hybrid graphs with no directed or semi-directed cycles (i.e., no edges point the same direction). 

\begin{definition}
A \textbf{hybrid graph (a.k.a., mixed graph)} is a graph consisting of directed and undirected edges. 
\end{definition}

\begin{definition}
A \textbf{collider} in a hybrid graph is a node with at least two parents. 
\end{definition}

\begin{definition}
Let $G_1 = (X, E_1)$ and $G_2 = (X, E_2)$ be two DAGs containing the same set of variables $X$. Then $G_1$ and $G_2$ are \textbf{Markov equivlent} if for every three mutually disjoint subsets $A, B, C \subseteq X$, $A$ and $B$ are d-separated by $C$ in $G_1$ if and only if $A$ and $B$ are d-separated by $C$ in $G_2$.
\end{definition}

\begin{proposition}
Let $G_1 = (X, E_1)$ and $G_2 = (X, E_2)$ be two DAGs containing the same set of variables $X$. Then the following are equivalent:
\begin{enumerate}
\item $G_1$ and $G_2$ are Markov equivalent. 
\item Based on the Markov condition, they entail the same conditional independencies. 
\item For every joint probability distribution $P$ of $X$, $<G_1, P>$ satisfies the Markov condition if and only if $<G_2, P>$ satisfies the Markov condition. 
\item They have the same skeleton and the same set of colliders. 
\end{enumerate}
\end{proposition}

\begin{definition}
An edge is \textbf{compelled} in an equivalent class $\mathcal{G}$ if it exists in every DAG in $\mathcal{G}$. 
\end{definition}

\begin{definition}
An equivalence class $\mathcal{G}$ of directed acyclic graphs over $X$ variables can be described by the \textbf{essential graph} \cite{andersson1997characterization} (or completed partially directed acyclic graph \textbf{CPDAG} \cite{chickering2002optimal}, \textbf{DAG pattern} \cite{neapolitan2004learning}) of $\mathcal{G}$ which is a hybrid graph $G^*$ over $X$ defined as follows: 
\begin{itemize}
\item $X_i \rightarrow X_j \in E(G^*)$ iff $X_i \rightarrow X_j \in E(G)$ for every $G \in \mathcal{G}$,
\item $X_i - X_j \in E(G^*)$ iff there are $G_1, G_2 \in \mathcal{G}$ such that $X_i \rightarrow X_j \in E(G_1)$ and  $X_i \leftarrow X_j \in E(G_2)$.
\end{itemize}
\end{definition}
This is a unique representation of an equivalent class. 

\begin{definition}
The \textbf{moral graph} of a DAG is the skeleton of the hybrid graph obtained by adding an undirected edge between each pair of non-adjacent parents that have a common child. 
\end{definition}

\bibliographystyle{named}
\bibliography{/home/kl/Documents/causal_discovery_ref_list}

\end{document}